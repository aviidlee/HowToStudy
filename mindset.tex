\chapter{Mindset}
	\cmt{
	Useful mental habits for educational success and how to cultivate them
	}
	\section{Overview}
	\paragraph{}
	Although there is not a single correct mindset, here are the traits that are commonly
	observed in academically-successful people. 
	
	\begin{enumerate}
		\item Taking responsibility for your own education
			\begin{enumerate}
			\item Always asking ``What can \emph{I} do?"
			\item Willingness to do what it takes, rather than relying on other people
			\end{enumerate}
		\item Genuine desire to learn (i.e., curiosity)
		\item Willingness to make mistakes
		\item Believing that you can do it 
		\item Perseverance and patience
	\end{enumerate}

	\section{Being responsible for your own education}
	\paragraph{}
	% What mindset is 
	Your mindset, simply put, is how you view the world - it is how you interpret 
	events that happen, how you think the world works, who or what you hold responsible
	for events, what you think your role in this world is, etc. It is your attitudes and 
	beliefs. 
	
	\paragraph{}
	% Why mindsets are important 
	Your mindset is the \emph{single} most important determinant of what and how
	much you learn. It is the foundation without which nothing else really matters, because 
	your beliefs about yourself and your education influence all of your thoughts and actions around
	how you study and learn. 
	
	\paragraph{}
	There is no single mindset that is the best for achieving educational success. However, there
	are some beliefs and attitudes which are very useful, and others which are very harmful. 
	Here is the single most important lesson of this chapter; if you can nail this, the rest of the 
	book may help you, but it won't be essential: 
	
	\begin{crit}You are responsible for your own education.\end{crit}
	
	\paragraph{}
	% Brief Explanation of what it means to be responsible for your own education
	Ultimately, it is \emph{your} responsibility to make sure that you learn. This is not to say
	that your teachers have no duty to you - they also have a responsibility to try their best to
	help you learn. However, the key here is \emph{help}. Nobody can \emph{make} you learn, or 
	magically pour knowledge into your head. Furthermore, this is the attitude that says you
	will learn \emph{regardless} of whether your teacher is any good or not, regardless of whether 
	you have access to good resources or favourable circumstances. The person who benefits most from
	being well-educated is \emph{you}, so you will do what it takes to learn. I re-iterate that this
	belief is not one which absolves your teachers or other people from their own duties, but rather
	one which empowers you - this is the attitude that says ``I want and need to, and I shall." 
	This is the attitude that says you will prevail over circumstance. 
	
	\paragraph{}
	% Concrete examples 
	Concretely, what does it mean to be responsible for your education? It means that you set 
	a goal and do everything you can to achieve it. It means that when challenges arise and 
	unfavourable things happen, you ask ``how can I overcome this?" rather than ``it's all their
	fault." It means that although you ask for help, you always take the initiative and first step
	to solving your problems, and find new avenues when old ones are blocked. 
	
		\begin{eg}{Having a constructive attitude}
			\begin{itemize}
				\item Alice doesn't understand the lectures. She goes over them again, making notes
				and writing down specific questions about the concepts presented. She searches her
				textbook and online to help her understand, and talks to her lecturer about her
				questions. 
				
				\item Bob's maths teacher has no idea what he's talking about. When Bob asks 
				questions, he gives incomprehensible and contradictory answers. Bob does his best
				to learn from textbooks, asks his friends, asks other teachers, and goes online to
				learn from places like Khan academy. 
				
				\item Carol doesn't have money to buy the required chemistry textbook. She borrows
				it from the library for the time being, and forgoes a couple of trips to the cinema
				to help her afford it. 
				
			\end{itemize}
		\end{eg}

	\begin{exercise}
	Think of some recent challenges you have had with learning. What did you do to overcome them?
	\end{exercise}	

	