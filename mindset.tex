\chapter{Mindset}
	\cmt{
	Useful mental habits for educational success and how to cultivate them
	}
	\section{Overview}
	\paragraph{}
	Although there is not a single correct mindset, here are the traits that are commonly
	observed in academically-successful people. 
	
	\begin{enumerate}
		\item Taking responsibility for your own education
			\begin{enumerate}
			\item Always asking ``What can \emph{I} do?"
			\item Willingness to do what it takes, rather than relying on other people
			\end{enumerate}
		\item Genuine desire to learn (i.e., curiosity)
		\item Willingness to make mistakes
		\item Believing that you can do it 
		\item Perseverance and patience
	\end{enumerate}

	\section{Be responsible for your own education}
	\paragraph{}
	% What mindset is 
	Your mindset, simply put, is how you view the world - it is how you interpret 
	events that happen, how you think the world works, who or what you hold responsible
	for events, what you think your role in this world is, etc. It is your attitudes and 
	beliefs. 
	
	\paragraph{}
	% Why mindsets are important 
	Your mindset is the \emph{single} most important determinant of what and how
	much you learn. It is the foundation without which nothing else really matters, because 
	your beliefs about yourself and your education influence all of your thoughts and actions around
	how you study and learn. 
	
	\paragraph{}
	There is no single mindset that is the best for achieving educational success. However, there
	are some beliefs and attitudes which are very useful, and others which are very harmful. 
	Here is the single most important lesson of this chapter; if you can nail this, the rest of the 
	book may help you, but it won't be essential: 
	
	\begin{crit}You are responsible for your own education.\end{crit}
	
	\paragraph{}
	% Brief Explanation of what it means to be responsible for your own education
	Ultimately, it is \emph{your} responsibility to make sure that you learn. This is not to say
	that your teachers have no duty to you - they also have a responsibility to try their best to
	help you learn. However, the key here is \emph{help}. Nobody can \emph{make} you learn, or 
	magically pour knowledge into your head. Furthermore, this is the attitude that says you
	will learn \emph{regardless} of whether your teacher is any good or not, regardless of whether 
	you have access to good resources or favourable circumstances. The person who benefits most from
	being well-educated is \emph{you}, so you will do what it takes to learn. I re-iterate that this
	belief is not one which absolves your teachers or other people from their own duties, but rather
	one which empowers you - this is the attitude that says ``I want and need to, and I shall." 
	This is the attitude that says you will prevail over circumstance. 
	
	\paragraph{}
	% Concrete examples 
	Concretely, what does it mean to be responsible for your education? It means that you set 
	a goal and do everything you can to achieve it. It means that when challenges arise and 
	unfavourable things happen, you ask ``how can I overcome this?" rather than ``it's all their
	fault." It means that although you ask for help, you always take the initiative and first step
	to solving your problems, and find new avenues when old ones are blocked. 
	
	\paragraph{}
	I have heard students complain that they could not learn because their teacher's style 
	was not to their liking, or because the teacher ``doesn't know anything". Although it 
	is very helpful to have a knowledgeable teacher, you are not at the mercy of educators.
	As long as you are willing to put in some work, you can always find some other way of learning
	- books, other teachers, online courses, even your friends. 

	\begin{exercise}
	Choose one challenge to your learning, whether that is a badly-structured course, 
	a terrible lecturer, not enough money to buy textbooks, or a noisy neighbour who 
	won't let you study in peace. Write down 3 ideas for how you can overcome the problem.
	\end{exercise}

	\subsection{Be active in your education}
	\paragraph{}
	A direct consequence for taking responsibility for your own education is that you 
	become an \textbf{active learner}: you actually \emph{do} something to learn, 
	and always seek out learning, rather than expecting to have it poured into your
	head by lecturers and tutors. It means that you always put effort in to try to 
	understand concepts and solve problems.
	
	\paragraph{}
	For example, I had a student who asked help with Taylor series. Such a request is
	good - it's one step up from the student who turns up to a tutorial without having
	given any thought to what they should learn. However, when the tutorial started and
	I asked the student what about Taylor series was confusing, she replied ``everything". 
	She hadn't even read the definition of Taylor series, although it was given in lectures,
	is in the textbook and workbook for the course, and is accompanied by several examples.
	If you don't even \emph{try} to understand something, then you will never be able to learn.
	I could give you a whole book of similar examples, from students who expect you to help
	them with assignment questions without having made a serious attempt, to students who turn
	up to Q\&A tutorials without any questions. 
	
	\paragraph{}
	To be clear, asking questions and asking for help is \emph{essential} - but you must first make
	some attempt on your part and ask questions that are as \emph{specific} as you can 
	manage. This will help you learn more, and it will help the tutor help you, because 
	they will have a better idea about what is confusing you, rather than trying to guess
	from experience. 
	
	\paragraph{}
	Although asking any question is far better than not asking any, some questions are more
	effective for your learning than others. As mentioned above, specificity is one of the characteristics
	of effective questions. Here are some examples of ``bad" questions, and their improved counterparts. We will talk more about how to ask effective questions later. 
	
	\newcolumntype{C}[1]{>{\centering\let\newline\\\arraybackslash\hspace{0pt}}m{#1}}
	\begin{tabular}{|c | m{5cm} |}
		\hline
		``Bad" & ``Better" \\
		\hline
		What do we need to study for the exam? & Does the exam place a particular emphasis on any topic? 
		How is topic X assessed? \\
		\hline
		Can you do question 5 ? & I've tried question 5 and I got stuck at this step because <reason>; where did I go wrong? \\
		\hline
		What the hell is question 3 asking? & I don't understand which method the question is asking for; does it want us to differentiate using first principles, or can we use the chain rule?
		\\
		\hline
	\end{tabular}

	\begin{exercise}
	Write down 3 questions you will ask at your next tutorial/recitation/lesson. 
	If you have no questions or they are very vague, review the latest material 
	that you've covered by reading the lecture slides, relevant sections of the 
	textbook, and/or solving related problems. 
	\end{exercise}

	\paragraph{}
	The second major part of being an active learner is regularly asking yourself
	``what do I need to learn? What do I need to improve?" Again, the more specific
	your answer is, the better. We will talk more about how to answer these questions 
	effectively in later chapters, but if you are not already in the habit of asking
	these questions, simply asking the question is a big improvement. To make this 
	regular, decide on some frequency - after every lesson, at the end of every week,
	etc. To make sure you don't forget, set an alarm or put it in your calendar. 
	
	\paragraph{}
	Regularly asking how you can improve will well... help you to improve. One of the
	reasons that students find it difficult to study is that they don't always have
	a clear idea of what exactly they should study - they just feel like they have a 
	whole semester of content to learn. Asking how you can improve will also help you
	to ask more effective questions to your teachers. 
	

	
	
	
	